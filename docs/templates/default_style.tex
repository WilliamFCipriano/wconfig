% Document Shell - Header

\documentclass{article}
\usepackage{titlesec}
\usepackage[utf8]{inputenc}
\usepackage{listings}
\usepackage{xcolor}

\usepackage{color}
\definecolor{deepblue}{rgb}{0,0,0.5}
\definecolor{deepred}{rgb}{0.1,0,0}
\definecolor{deepgreen}{rgb}{0,0.5,0}
\definecolor{background}{rgb}{.9,.9,.9}

\usepackage[scaled]{helvet}
\renewcommand\familydefault{\sfdefault} 
\usepackage[T1]{fontenc}

\DeclareFixedFont{\ttb}{T1}{txtt}{bx}{n}{12}
\DeclareFixedFont{\ttm}{T1}{txtt}{m}{n}{12}

\definecolor{light-gray}{gray}{0.95}
\newcommand{\monospaced}[1]{\colorbox{light-gray}{\texttt{#1}}}

\newcommand\pythonstyle{\lstset{
language=Python,
xleftmargin = 2cm,
framexleftmargin = 1em,
basicstyle=\ttm,
otherkeywords={self, assert},   
keywordstyle=\ttm\color{deepblue},
emph={Exception}
emphstyle=\color{deepred}, 
stringstyle=\color{deepgreen},
showstringspaces=false
}}

\lstnewenvironment{python}[1][]
{
\pythonstyle
\lstset{#1}
}
{}

\setcounter{secnumdepth}{4}

\titleformat{\paragraph}
{\normalfont\normalsize\bfseries}{\theparagraph}{1em}{}
\titlespacing*{\paragraph}
{0pt}{3.25ex plus 1ex minus .2ex}{1.5ex plus .2ex}

\begin{document}

% Document Shell - Header End

% Title Page

\begin{titlepage}
   \vspace*{\stretch{1.0}}
   \begin{center}
      \Large\textbf{Project Name - Developer Manual}\\
      \large\textit{William F. Cipriano}\\
      \tiny{the definitive edition}
   \end{center}
   \vspace*{\stretch{2.0}}
\end{titlepage}

% Title Page End

% Table of Contents

\tableofcontents

\listoftables


\newpage

% Table of Contents End

% Body

\section{Example}

\begin{table}[h!]
\centering
\begin{tabular}{||c c c c||} 
 \hline
 Col1 & Col2 & Col2 & Col3 \\ [0.5ex] 
 \hline\hline
 1 & 6 & 87837 & 787 \\ 
 2 & 7 & 78 & 5415 \\
 3 & 545 & 778 & 7507 \\
 4 & 545 & 18744 & 7560 \\
 5 & 88 & 788 & 6344 \\ [1ex] 
 \hline
\end{tabular}
\caption{Table to test captions and labels}
\label{table:1}
\end{table}


\subsection{Text Modes}
\huge big font \\
\small small font \\
\monospaced{monospaced font} \\

\subsubsection{Lists}
\begin{enumerate}
   \item The labels consists of sequential numbers.
   \begin{itemize}
     \item The individual entries are indicated with a black dot, a so-called bullet.
     \item The text in the entries may be of any length.
   \end{itemize}
   \item The numbers starts at 1 with every call to the enumerate environment.
\end{enumerate}
\paragraph{Code}

\begin{python}
class MyClass(Yourclass):
    def __init__(self, my, yours, stuff=here):
        bla = '5 1 2 3 4'
        print(bla)
        assert bla != '23'
        raise Exception
\end{python}

\newpage


% Body End

% Bibliography

\section{bibliography}

\begin{thebibliography}{9}
\bibitem{latexcompanion} 
Michel Goossens, Frank Mittelbach, and Alexander Samarin. 
\textit{The \LaTeX\ Companion}. 
Addison-Wesley, Reading, Massachusetts, 1993.
 
\bibitem{einstein} 
Albert Einstein. 
\textit{Zur Elektrodynamik bewegter K{\"o}rper}. (German) 
[\textit{On the electrodynamics of moving bodies}]. 
Annalen der Physik, 322(10):891–921, 1905.
 
\bibitem{knuthwebsite} 
Knuth: Computers and Typesetting,
\\\texttt{http://www-cs-faculty.stanford.edu/\~{}uno/abcde.html}
\end{thebibliography}

\end{document}

% Body End


